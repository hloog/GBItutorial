\begin{frame}{Vorbereitung}
	\framesubtitle{mod und div}
	mod = Rest bei einer Ganzzahldivision, vergleichbar mit dem \%-Operator in Java\\
	div = Ergebnis einer Ganzzahldivision, also ohne Rest\\
	Beispiel:\\
	\center
	\begin{tabular}{c|c}
		\textbf{mod}   & \textbf{div}  \\
		\hline
		$18$mod$4=2$   & $18$div$4=4$  \\
		$19$mod$4=3$   & $19$div$4=4$  \\
		$20$mod$4=0$   & $20$div$4=5$  \\
		$37$mod$13=11$ & $37$div$13=2$ \\
	\end{tabular}
	\flushleft
	\emph{Rechenregeln}\\
	\small
	\centerline{$\forall x,y\in\mathbb{N}_{0}:x=y\cdot(x$div$y)+(x$mod$y)$}
	\centerline{$\forall x,y\in\mathbb{N}_{0}:(x\pm y)$mod$k=((x$mod$k)\pm(y$mod$k))$mod$k$}
	\normalsize
\end{frame}

\begin{frame}{Zahlendarstellung}
	Alphabet $Z_{n}=\{x\in\mathbb{N}_{0}\vert x<n\}$\\
	Funktion $Num_{k}: Z_{k}^{\ast}\longrightarrow\mathbb{N}_{0},$\\
	\small
	$Num_{k}(\epsilon)=0$\\
	$Num_{k}(x_{m}x_{m-1}\dots x_{3}x_{2}x_{1})=k\cdot Num_{k}(x_{m}x_{m-1}\dots x_{3}x_{2})+num_{k}(x_{1})$\\
	Man kürzt $Num_{k}(x)$ oft direkt zu $num_{k}(x)$, wenn $x$ in $Z_{k}$ liegt\\
	\normalsize
	Funktion $num_{k}: Z_{k}\longrightarrow\mathbb{N}_{0},$\small$\quad num_{k}(x)=x$\normalsize\\
\end{frame}

\begin{frame}{Zahlendarstellung}
	\framesubtitle{Beispiele}
	\begin{enumerate}
		\item[$\triangleright$] $Num_{10}(742)=10\cdot Num_{10}(74)+num_{10}(2)$\\
		      $=10\cdot(10\cdot Num_{10}(7)+num_{10}(4))+2$\\
		      $=10\cdot(10\cdot(10\cdot (Num_{10}(\epsilon))+num_{10}(7))+4)+2$\\%hier ausführlich mit epsilon, unten mit abkürzen
		      $=10\cdot(10\cdot(10\cdot (0)+7)+4)+2$\\
		      $=10\cdot(10\cdot(7)+4)+2$\\
		      $=10\cdot(74)+2$\\
		      $=742$\\
		\item[$\triangleright$] $Num_{2}(01101)=2\cdot Num_{2}(0110)+num_{2}(1)$\\
		      $=2\cdot(2\cdot Num_{2}(011)+num_{2}(0))+2^{0}\cdot1$\\
		      $=2\cdot(2\cdot (2\cdot Num_{2}(01)+num_{2}(1))+0)+2^{0}\cdot1$\\
		      $=2\cdot(2\cdot (2\cdot (2\cdot Num_{2}(0)+num_{2}(1))+1)+0)+2^{0}\cdot1$\\
		      $=2^{2}\cdot (2\cdot (2\cdot num_{2}(0)+1)+1)+2^{1}\cdot0+2^{0}\cdot1$\\
		      $=2^{3}\cdot (2\cdot 0+1)+2^{2}\cdot1+2^{1}\cdot0+2^{0}\cdot1$\\
		      $=2^{4}\cdot 0+2^{3}\cdot 1+2^{2}\cdot1+2^{1}\cdot0+2^{0}\cdot1$\\
		      $=0+8+4+0+1=13$
	\end{enumerate}
\end{frame}

\begin{frame}{Hexadezimaldarstellung}
	Wie stellt man Basen $\geq10$ dar?\\
	$Z_{16}=\{0,1,2,3,4,5,6,7,8,9,10,11,12,13,14,15\}$?\\
	Nein, denn $10,11,12,13,14,15$ bestehen bereits aus zwei Zeichen, man weicht auf andere zeichen aus\\
	$Z_{16}=\{0,1,2,3,4,5,6,7,8,9,A,B,C,D,E,F\}$\\
	wobei $A=10,B=11,\dots$\\
\end{frame}

\begin{frame}{Zahlenrepräsentation}
	Sei $k\in\mathbb{N}_{0}$ und $k\geq2$\\
	$Repr_{k}:\mathbb{N}_{0}\longrightarrow Z_{k},$\\
	$n\mapsto~
		\begin{cases}
			repr_{k}(n)                                        & \text{,falls $n<k$}     \\
			Repr_{k}(n\text{div}k)\cdot repr_{k}(n\text{mod}k) & \text{,falls $n\geq k$} \\
		\end{cases}$\\
	\fbox{$\forall n\in\mathbb{N}_{0}:Num_{k}(Repr_{k}(n))=n$}\\
\end{frame}

\ifdefined\showSolutions
	\begin{frame}{Aufgaben}
	Geben Sie an:
	\begin{description}
		\item[Aufgabe 1] $Num_{5}(43210_{10})$
		\item[Aufgabe 2] $Repr_{2}(126_{10})$
		\item[Aufgabe 3] $Repr_{16}(900_{10})$
	\end{description}
\end{frame}

	\begin{frame}{Lösungen}
	\small
	\begin{description}
		\item[Aufgabe 1] $Num_{5}(43210_{10})$\\
		      \pause
		      $=4\cdot 5^{4}+3\cdot 5^{3}+2\cdot 5^{2}+1\cdot 5^{1}+0\cdot 5^{0}$\\
		      \pause
		      $=2500+375+50+5=2930$
	\end{description}
\end{frame}

\begin{frame}{Lösungen}
	\small
	\begin{description}
		\item[Aufgabe 2] $Repr_{2}(126_{10})$\\
		      \pause
		      $=Repr_{2}(126$div$2)\cdot repr_{2}(126$mod$2)$\\
		      \pause
		      $=Repr_{2}(63)\cdot repr_{2}(0)$\\
		      \pause
		      $=Repr_{2}(63$div$2)\cdot repr_{2}(63$mod$2)\cdot 0$\\
		      \pause
		      $=Repr_{2}(31)\cdot repr_{2}(1)\cdot 0$\\
		      \pause
		      $=Repr_{2}(31$div$2)\cdot repr_{2}(31$mod$2)\cdot 10$\\
		      \pause
		      $=Repr_{2}(15)\cdot repr_{2}(1)\cdot 10$\\
		      \pause
		      $=Repr_{2}(15$div$2)\cdot repr_{2}(15$mod$2)\cdot 110$\\
		      \pause
		      $=Repr_{2}(7)\cdot repr_{2}(1)\cdot 110$\\
		      \pause
		      $=Repr_{2}(7$div$2)\cdot repr_{2}(7$mod$2)\cdot 1110$\\
		      \pause
		      $=Repr_{2}(3)\cdot repr_{2}(1)\cdot 1110$\\
		      \pause
		      $=Repr_{2}(3$div$2)\cdot repr_{2}(3$mod$2)\cdot 11110$\\
		      \pause
		      $=Repr_{2}(1)\cdot repr_{2}(1)\cdot 11110$\\
		      \pause
		      $=Repr_{2}(1$div$2)\cdot repr_{2}(1$mod$2)\cdot 111110$\\
		      \pause
		      $=Repr_{2}(0)\cdot repr_{2}(1)\cdot 111110$\\
		      \pause
		      $=1111110_{2}$\\
	\end{description}
\end{frame}

\begin{frame}{Lösungen}
	\small
	\begin{description}
		\item[Aufgabe 3] $Repr_{16}(900_{10})$\\
		      \pause
		      $=Repr_{16}(900$div$16)\cdot repr_{16}(900$mod$16)$\\
		      \pause
		      $=Repr_{16}(56)\cdot repr_{16}(4)$\\
		      \pause
		      $=Repr_{16}(56$div$16)\cdot repr_{16}(56$mod$16)\cdot 4$\\
		      \pause
		      $=Repr_{16}(3)\cdot repr_{16}(8)\cdot 4$\\
		      \pause
		      $=Repr_{1}(3$div$16)\cdot repr_{16}(3$mod$16)\cdot 84$\\
		      \pause
		      $=repr_{16}(0)\cdot repr_{16}(3)\cdot 84$\\
		      \pause
		      $=384_{16}$\\
	\end{description}
\end{frame}

\else
	\begin{frame}{Aufgaben}
	Geben Sie an:
	\begin{description}
		\item[Aufgabe 1] $Num_{5}(43210_{10})$
		\item[Aufgabe 2] $Repr_{2}(126_{10})$
		\item[Aufgabe 3] $Repr_{16}(900_{10})$
	\end{description}
\end{frame}

\fi

\begin{frame}{Übersetzung von Zahldarstellungen}
	\framesubtitle{Kommt wieder im zweiten Semester in Digitaltechnik \& Entwurfsverfahren}
	Geht nur, wenn Basis Vielfaches von anderer ist, zB 2 von 16\\
	Beispiel: Zahlen von $Z_{2}$ nach $Z_{16}$ und zurück konvertieren. $2^{\textcolor{red}{4}}=16$\\
	Wir müssen \textcolor{red}{4} Zahlen aus der Dualzahl zusammenfassen, um die Hexadezimalzahl zu bekommen\\
	$1 0110 0101 1010_{2}=\textcolor{blue}{000}1\thinspace0110\thinspace0101\thinspace1010_{2}=165A_{16}$\par
	\begin{tikzpicture}
		\hspace{31.7mm}
		\draw[decoration={brace,mirror,raise=-4pt},decorate] (0,0) -- node[below=-2pt] {$1$} (0.75,0);
		\hspace{8.35mm}
		\draw[decoration={brace,mirror,raise=-4pt},decorate] (0,0)--node[below=-2pt]{$6$}(0.75,0);
		\hspace{8.355mm}
		\draw[decoration={brace,mirror,raise=-4pt},decorate] (0,0)--node[below=-2pt]{$5$}(0.75,0);
		\hspace{8.4mm}
		\draw[decoration={brace,mirror,raise=-4pt},decorate] (0,0)--node[below=-2pt]{$10$}(0.75,0);
	\end{tikzpicture}
	\textcolor{white}{.}\\%schöner machen!
	$C7F_{16}=1100\thinspace 0111\thinspace 1111_{2}$\\
	\begin{tikzpicture}
		\hspace{15.57mm}
		\draw[decoration={brace,mirror,raise=-4pt},decorate] (0,0) -- node[below=-2pt] {{\footnotesize$12$}} (0.75,0);
		\hspace{8.35mm}
		\draw[decoration={brace,mirror,raise=-4pt},decorate] (0,0)--node[below=-2pt]{{\footnotesize$7$}}(0.75,0);
		\hspace{8.355mm}
		\draw[decoration={brace,mirror,raise=-4pt},decorate] (0,0)--node[below=-2pt]{{\footnotesize $15$}}(0.75,0);
	\end{tikzpicture}
\end{frame}
%Rechnen in Zk (restklassen) mit zeichnung. ``vergleichbar mit restklassen aus LA1''
%Homomorphismen...
\begin{frame}{Huffmann-Codierung}
	gegeben: Alphabet $A$ und Wort $W\in A^{\ast}$\\
	\begin{itemize}
		\item[$\bullet$] $h:A^{\ast}\longrightarrow Z_{2}^{\ast}$
		\item[$\bullet$] $\epsilon$-freier Homomorphismus
	\end{itemize}
	findet kürzeste Codierung von $w$ in $Z_{2}$\\
\end{frame}
\begin{frame}
	\frametitle{Huffman-Codierung}
	\framesubtitle{Huffman-Baum}

\end{frame}
