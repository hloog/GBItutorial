\documentclass{beamer}
\usepackage[T1]{fontenc}
\usepackage[ngerman]{babel}
\usepackage[utf8]{inputenc}
\usepackage{lmodern}
\usepackage{amsmath}
\usepackage{amssymb}
\usepackage{mathrsfs}
\usepackage{color}
\usepackage{latexsym}%for \Box and \leadsto
\usepackage{mathtools} % for xRightarrow
\usepackage{tikz}%zum zusammenfassen
\usetikzlibrary{decorations.pathreplacing}%zum zusammenfassen
%
\title{Kapitel 7: Formale Sprachen}
\author{Jannis Weis, Raphael Adalid Braun}
\date{\today}
\institute{KIT - Karlruher Institut für Technologie}
\usetheme{Warsaw}
\begin{document}
	\begin{frame}
		\titlepage
	\end{frame}
	\begin{frame}
		\frametitle{Wiederholung}
		Formale Sprache $L$ über Alphabet $A^{\ast}: \quad L\subseteq A^{\ast}$\\
		Alle Wörter in $L$ sind konkatenierte Zeichen aus $A$\\
		Meist induktiv oder formal definiert, zB:\\
		Sei $A=\{a,b\}$ ein Alphabet und $L$ die Sprache über $A^{\ast}$, in der die Anzahl der a's und b's pro Wort gleich ist.\\
		induktiv:
		\centerline{$L_{0}=\{\epsilon\}$}
		\centerline{$\forall n\in\mathbb{N}_{0}:L_{n+1}=\{abw,awb,wab,baw,bwa,wba\vert w\in L_{n}\}$}
		\centerline{$L=\bigcup\limits_{i\in\mathbb{N}_{0}}L_{i}$}
		formal:
		\centerline{$L=\{w\in A^{\ast}\vert N_{a}(w)=N_{b}(w)\}$}
	\end{frame}
	\begin{frame}
		\frametitle{Produkt/Konkatenation formaler Sprachen}
		Sprachen $L_{1}$ und $L_{2}$ konkateniert:\\
		\centerline{$L_{1}\cdot L_{2}=\{w_{1}\cdot w_{2}\vert w_{1}\in L_{1}\wedge w_{2}\in L_{2}\}$}
		Beispiel:\\
		Seien $L_{1}=\{aa, aaa\}$ und $L_{2}=\{b,c\}$ formale Sprachen, dann:\\
		$L_{1}\cdot L_{2}=\{aa,aaa\}\cdot\{b,c\}$\\
		\hspace{11.5mm}$=\{aa\cdot b,aa\cdot c,aaa\cdot b,aaa\cdot c\}$\\
		\hspace{11.5mm}$=\{aab,aac,aaab,aaac\}$\\
		\textcolor{red}{Potenzen von formalen Sprachen}\\
		$L\cdot L\cdot L\cdot L=L^{4},\quad\lvert L^{n}\rvert=\lvert L\rvert^{n}\thickspace(n\in\mathbb{N}_{0})$\\
	\end{frame}
	%Konkatenationsabschluss?
	\begin{frame}
		\frametitle{Darstellung von formalen Sprachen}
		\framesubtitle{Vorbereitung auf reguläre Ausdrücke}
		Alphabet $A$ mit Zeichen in $A$ gegeben,\\
		\begin{itemize}
			\item Klammern erlaubt
			\item Zeichen aus $A$ und $\epsilon$ erlaubt
			\item Komma erlaubt
			\item $+$ mind. ein
			\item $\ast$ beliebig viele ($0$ bis $\infty$)
		\end{itemize}
		Beispiel:\\
		$L=\{a^{k}b^{j}\vert k\in\mathbb{N}_{0}\wedge j\in\mathbb{N}\}$\\
		$L=\{\{a\}^{\ast}\cdot\{b\}^{+}\}$\\
	\end{frame}
	\begin{frame}
		\frametitle{Aufgaben}
		Sei $A=\{a,b\}$ ein Alphabet\\
		Gib die Menge $M$ aller Wörter über $A$ an, die \dots\\
		\begin{description}
			\item[Aufgabe 1] das Teilwort \frqq ab\flqq enthalten
			\item[Aufgabe 2] deren vorletztes Zeichen ein \frqq b\flqq ist
			\item[Aufgabe 3] das Teilwort \frqq bb\flqq \thinspace \underline{nicht} enthalten
		\end{description}
		\fbox{\quad$\{$\quad$\}$\quad$a$\quad$b$\quad$\epsilon$\quad$\ast$\quad$+$\quad,\quad}\\
	\end{frame}
	\begin{frame}
		\frametitle{Lösungen}
		\begin{description}
			\item[Aufgabe 1] das Teilwort \frqq ab\flqq enthalten
			\item[Aufgabe 2] deren vorletztes Zeichen ein \frqq b\flqq ist
			\item[Aufgabe 3] das Teilwort \frqq bb\flqq \thinspace \underline{nicht} enthalten
		\end{description}
		\begin{description}
			\pause
			\item[Lösung 1] $M=\{\{a,b\}^{\ast}\cdot\{ab\}\cdot\{a,b\}^{\ast}\}$
			\pause
			\item[Lösung 2] $M=\{\{a,b\}^{\ast}\cdot\{b\}\cdot\{a,b\}\}$
			\pause
			\item[Lösung 3] $M=\{\{a,ba\}^{\ast}\cdot\{b,\epsilon\}\}$
		\end{description}
	\end{frame}
\end{document}