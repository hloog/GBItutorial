\documentclass[18pt]{beamer}
\usepackage{templates/beamerthemekit}
\usepackage[T1]{fontenc}
\usepackage[utf8]{inputenc}
\usepackage{libertine}
\usepackage{amsmath}
\usepackage{amssymb}
\usepackage{mathrsfs}
\usepackage{latexsym}%for \Box and \leadsto
\usepackage{mathtools} % for xRightarrow
\usepackage{tikz}%zum zusammenfassen
\usepackage{xifthen}
\newcommand{\inserttask}[2][]{\title[Aufgabe]{} %zum einfügen von Aufgaben
 \ifdefined\showSolutions
  \input{../../tasks/#2}\ifthenelse{\isempty{#1}}{}{\input{../../solutions/#1}}
 \else\input{../../tasks/#2}\fi
}
\usetikzlibrary{decorations.pathreplacing}%zum zusammenfassen

\title[Aussagenlogik]{Kapitel 5: Aussagenlogik}
\subtitle{Tutorium - } %Tutoriums-Nr einfügen
\author{Raphael Adalid Braun, Jannis Weis}
\institute{Grundbegriffe der Informatik | WS 2018/19}
\titlelogo{no_logo}

\begin{document}
\selectlanguage{ngerman}

\begin{frame}
 \titlepage
\end{frame}

%Entfernen für Tutoriumsfolien-%
\def\showSolutions{1}					 %
%------------------------------%

\section{Aussagenlogik}
\title[Aussagenlogik]{}
\begin{frame}{Aussagenlogik}
 \underline{Zweiwertigkeit:} Jede Aussage ist entweder wahr oder falsch\\
 \underline{Aussagenlogische Konnektive}\\
 Negation: $\neg$\\
 logisches \textbf{und}: $\wedge$\\
 logisches \textbf{oder}: $\vee$\\
 logische Folgerung: $\rightarrow$ (Implikation)\\
 Aussagevariablen: $Var_{AL}$, Elemente meist mit $P, Q, R, S$ bezeichnet
 Alphabet der Aussagenlogik: $A_{AL}=\{(,),\neg,\wedge,\vee,\rightarrow\}\cup Var_{AL}$\\
\end{frame}

\begin{frame}{Konstruktionsabbildungen}
 $f_{\neg}: A^{\ast}_{AL}\longrightarrow A^{\ast}_{AL}, G\mapsto (\neg G)$\\
 $f_{\wedge}:A^{\ast}_{AL}\times A^{\ast}_{AL}\longrightarrow A^{\ast}_{AL}, (G, H)\mapsto (G\wedge H)$\\
 $f_{\vee}:A^{\ast}_{AL}\times A^{\ast}_{AL}\longrightarrow A^{\ast}_{AL}, (G, H)\mapsto (G\vee H)$\\
 $f_{\rightarrow}:A^{\ast}_{AL}\times A^{\ast}_{AL}\longrightarrow A^{\ast}_{AL}, (G, H)\mapsto (G\rightarrow H)$\\
 Bindungsstärke:\\
 $\neg>\vee>\wedge>\rightarrow$\\
\end{frame}

\begin{frame}{Aussagenlogischen Term konstruieren}
 Gesucht: $((A\rightarrow B)\wedge (\neg A\vee\neg C))$\\
 Vorgehen: Immer zuerst kleine Terme konstruieren und verbinden\\
 \textcolor{green}{$f_{\neg} ($}$A$\textcolor{green}{$)$}$ = \neg A$\\
 \textcolor{blue}{$f_{\neg}($}$C$\textcolor{blue}{$)$}$ = \neg C$\\
 \textcolor{red}{$f_{\vee}($}\textcolor{green}{$f_{\neg} ($}$A$\textcolor{green}{$)$}, \textcolor{blue}{$f_{\neg}($}$C$\textcolor{blue}{$)$}\textcolor{red}{$)$}$ = (\neg A\vee\neg C)$\\
 \textcolor{purple}{$f_{\rightarrow}($}$A,B$\textcolor{purple}{$)$}$ = (A\rightarrow B)$\\
 \textcolor{brown}{$f_{\rightarrow}($}\textcolor{red}{$f_{\vee}($}\textcolor{green}{$f_{\neg} ($}$A$\textcolor{green}{$)$}, \textcolor{blue}{$f_{\neg}($}$C$\textcolor{blue}{$)$}\textcolor{red}{$)$},\textcolor{purple}{$f_{\rightarrow}($}$A,B$\textcolor{purple}{$)$}\textcolor{brown}{$)$}$ = ((A\rightarrow B)\wedge (\neg A\vee\neg C))$\\
\end{frame}

\begin{frame}{Interpretation von Formeln}
 Interpretation $I$, $I: Var_{AL}\longrightarrow\mathbb{B}$\\
 Auswertung: $val_{I}: For_{AL}\longrightarrow\mathbb{B}$\\
 Funktioniert analog wie Konstruktionsabbildungen.\\
 Bsp: Seien $P, Q, R$ Aussagen mit $I(P), I(Q) = w$ und $I(R) = f$\\
 $val_{I}((((P\rightarrow Q)\rightarrow R)\wedge(R\vee Q))\rightarrow P$)\\ %boolsche funktion noch einbringen
 $=b_{\rightarrow}(val_{I}(((P\rightarrow Q)\rightarrow R)\wedge(R\vee Q)), val_{I}(P))$\\
 $=b_{\rightarrow}(b_{\wedge}(val_{I}((P\rightarrow Q)\rightarrow R),val_{I}(R\vee Q)), I(P))$\\
 $=b_{\rightarrow}(b_{\wedge}(b_{\rightarrow}(val_{I}(P\rightarrow Q), val_{I}(R)), val_{I}(R\vee Q)), w)$\\
 $=~b_{\rightarrow}(b_{\wedge}(b_{\rightarrow}(b_{\rightarrow}(val_{I}(P), val_{I}(Q)), I(R)), b_{\vee}(val_{I}(R), val_{I}(Q))), w)$\\
 $=b_{\rightarrow}(b_{\wedge}(b_{\rightarrow}(b_{\rightarrow}(I(P), I(Q)), f), b_{\vee}(I(R), I(Q))), w)$\\
 $=b_{\rightarrow}(b_{\wedge}(b_{\rightarrow}(b_{\rightarrow}(w, w), f), b_{\vee}(f, w)), w)$\\
 $=b_{\rightarrow}(b_{\wedge}(b_{\rightarrow}(w, f), w), w)$\\
 $=b_{\rightarrow}(b_{\wedge}(f, w), w)$\\
 $=b_{\rightarrow}(f, w)$\\
 $=w$\\
\end{frame}

\begin{frame}{Wahrheitstabellen}
 Eine Formel für alle möglichen (oder auch nur eine) Wahrheitsverteilungen auswerten\\
 Bsp: $F = (A\wedge B)\vee(B\rightarrow\neg A)$\\
 \textcolor{white}{.}\\ %zeilenumbruch/abstand schöner machen!
 \begin{tabular}{ccc|cc|c}
  $A$ & $\neg A$ & $B$ & $(A\wedge B)$ & $(B\rightarrow\neg A)$ & $F$ \\
  \hline
  $w$ & $f$      & $w$ & $w$           & $f$                    & $w$ \\
  $w$ & $f$      & $f$ & $f$           & $w$                    & $w$ \\
  $f$ & $w$      & $w$ & $f$           & $w$                    & $w$ \\
  $f$ & $w$      & $f$ & $f$           & $w$                    & $w$ \\
 \end{tabular}\\
 In diesem Fall ist $F$ sogar eine \emph{Tautologie}, d.h. alle Interpretationen von $F$ sind wahr.\\
 Zwei Formeln $G$ und $H$ heißen \emph{äquivalent}, wenn für alle Interpretationen gilt: $val_{I}(G) = val_{I}(H)$\\
 Schreibweise: $G\equiv H$\\
\end{frame}

\begin{frame}{Aussagenlogische Regeln}
 $F,G,H\in For_{AL}$\\
 $F\wedge F=F,F\vee F=F$\\
 $F\wedge G=G\wedge F, F\vee G=G\vee F$\\
 $(F\wedge G)\vee H=(F\vee H)\wedge(G\wedge H)$\\
 $(F\vee G)\wedge H=(F\wedge H)\vee(G\vee H)$\\
 $\neg(F\vee G)=(\neg F\wedge\neg G), \neg(F\wedge G)=(\neg F\vee\neg G)$\\
 $F\wedge(F\vee G)=F, F\vee(F\wedge G)=F$\\
\end{frame}
%evtl modelle reinmachen? dann nochmal allg tautologie, erfüllbarkeit
\begin{frame}{Aussagenkalkül}
 \textcolor{red}{Axiome}\\
 Seien $G,H,k\in For_{AL}$\\
 $Ax_{AL1} = (G\rightarrow(H\rightarrow G))$\\
 $Ax_{AL2} = (G\rightarrow(H\rightarrow K))\rightarrow((G\rightarrow H)\rightarrow(G\rightarrow K))$\\
 $Ax_{AL3} = (\neg H\rightarrow\neg G)\rightarrow((\neg H\rightarrow G)\rightarrow H)$\\
 \textcolor{red}{Schlussregel} \emph{Modus Ponens}\\
 \emph{Aus wahrem kann man nur wahres folgern}, d.h. seien $G, H$ und $G\rightarrow H$ gegeben und sei $G$ und $G\rightarrow H$ wahr, dann muss auch $H$ wahr sein\\
 $MP : \frac{G\rightarrow H\quad G}{H}$\\
\end{frame}

\inserttask[cpt5_pt1_L1]{cpt5_pt1_A1}
\inserttask[cpt5_pt1_L2]{cpt5_pt1_A2}
\inserttask[cpt5_pt1_L3]{cpt5_pt1_A3}


\end{document}
