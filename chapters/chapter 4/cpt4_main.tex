\documentclass[18pt]{beamer}
\usepackage{templates/beamerthemekit}
\usepackage[T1]{fontenc}
\usepackage[utf8]{inputenc}
\usepackage{libertine}
\usepackage{amsmath}
\usepackage{amssymb}
\usepackage{mathrsfs}
\usepackage{xifthen}
\newcommand{\inserttask}[2][]{\title[Aufgabe]{} %zum einfügen von Aufgaben
 \ifdefined\showSolutions
  \input{../../tasks/#2}\ifthenelse{\isempty{#1}}{}{\input{../../solutions/#1}}
 \else\input{../../tasks/#2}\fi
}
\usetikzlibrary{decorations.pathreplacing}%zum zusammenfassen

\title[Wörter]{Kapitel 4: Wörter}
\subtitle{Tutorium - } %Tutoriums-Nr einfügen
\author{Raphael Adalid Braun, Jannis Weis}
\institute{Grundbegriffe der Informatik | WS 2018/19}
\titlelogo{no_logo}

\begin{document}
\selectlanguage{ngerman}

\begin{frame}
 \titlepage
\end{frame}

%Entfernen für Tutoriumsfolien-%
\def\showSolutions{1}					 %
%------------------------------%

\section{Wörter}
\title[Wörter]{}
\begin{frame}{Wörter}
	Für den Rest des Kapitels sei A ein Alphabet
	Definition: Ein \emph{Wort} über einem Alphabet A ist eine Folge von Zeichen aus A\\
	$\mathbb{Z}_{n}=\{i\in\mathbb{N}_{0}\vert0\leq i\leq n\} (n\in\mathbb{N}_{0})$\\
	Bsp: $\mathbb{Z}_{4} =\{0,1,2,3\},\mathbb{Z}_{1}=\{0\},\mathbb{Z}_{0}=\emptyset$\\
	Länge eines Wortes = Anzahl der aneinandergefügten Zeichen, geschrieben $\lvert w\rvert$ für Wort $w$\\
\end{frame}

\begin{frame}{Das leere Wort}
	geschrieben: $\epsilon$\\
	Das leere Wort ist eine Abbildung!\\
	$\epsilon :\mathbb{Z}_{0}\longrightarrow\emptyset$\\
	$\lvert\epsilon\rvert = 0$\\
	$\epsilon\cdot w = w = w\cdot\epsilon$\\
\end{frame}

\begin{frame}{Mengen von Wörtern}
	Menge aller Wörter\\
	$A^{\ast}=$Menge aller Wörter über A\\
	Bsp: $A = \{a, b\}, A^{\ast}$ enthält $\epsilon, a, b, ab, ba, aa, bb, abbabbabaabba,\dots$\\
	$A^{n}=$Menge aller Wörter der Länge n\\
	Bsp: $A=\{a,b\}, A^{0}=\{\epsilon\},A^{1}=\{a,b\},A^{2}=\{aa,ab,ba,bb\}$\\
	$\Rightarrow A^{\ast}=A^{0}\cup A^{1}\cup A^{2}\cup A^{3}\cup\dots = \bigcup\limits_{i\in\mathbb{N}_0}A^{i}$\\%vereinigungen gehen auch mit wörtern
\end{frame}

\begin{frame}{Mehr zu Wörtern}
	Konkatenation, das \emph{verbinden} von Zeichen\\
	schrank$\cdot$schlüssel = schrankschlüssel.\\
	\underline{Nicht} kommutativ! eiswaffel $\neq$ waffeleis\\
	$\lvert w_{1}\cdot w_{2}\rvert =\lvert w_{1}\rvert + \lvert w_{2}\rvert$\\
	\underline{Assoziativ:}$\forall w_{1},w_{2},w_{3}\in A^{\ast}:(w_{1}\cdot w_{2})\cdot w_{3} = w_{1}\cdot (w_{2}\cdot w_{3})$\\
	Potenzen von Wörtern\\
	$w^{0}=\epsilon$\\
	$\forall n\in\mathbb{N}_{0}:w^{n+1}=w^{n}\cdot w$\\
	$\lvert w^{n}\rvert = n\ast \lvert w\rvert$\\
\end{frame}

\begin{frame}{Formale Sprachen}
	L sei eine formale Sprache, $L\subseteq A^{\ast}$\\
	Bsp: $A =\{0,1,\dots ,8,9\}$, daraus lässt sich die Darstellung der Dezimalzahlen als formale Sprache erstellen\\
	Meist induktiv definiert\\
\end{frame}

\begin{frame}{Aufgaben}
	\textcolor{red}{Aufgabe 1} $A=\{a,b\}$\\
	$L_{0}=\{\epsilon\}$\\
	$\forall n\in\mathbb{N}_{0}: L_{n+1}=L_{n}\cup \{awb\vert w\in L_{n}\}$\\
	Gib $L_{1}, L_{2}, L_{3}$ und eine nicht induktive Definition für $L_{n}$ an.%Ln ={a^{k}b^{k}|k in N0, k <= n}
\end{frame}



\end{document}
