\documentclass{beamer}
\usepackage[T1]{fontenc}
\usepackage[ngerman]{babel}
\usepackage[utf8]{inputenc}
\usepackage{lmodern}
\usepackage{amsmath}
\usepackage{amssymb}
\usepackage{mathrsfs}
\usepackage{color}
\begin{document}
	\begin{frame}
		\frametitle{Potenzmengen}
		Mengen können Elemente von Mengen sein\\
		$M = \{1, \{2, 3\}, 4, 5\} $Menge mit $\lvert M\rvert = 4$\\
		\textcolor{red}{Wichtig:} $\lvert \emptyset\rvert = 0$, aber $\lvert\{\emptyset\}\rvert = 1$\\
		Potenzmenge von M: Schreibweise: $2^M $oder $\mathfrak{P}(M)$\\
		$\mathfrak{P}(M) = \{ A\vert A\subseteq M\}$(ugs. alle Teilmengen von M)\\
		Bsp: $M = \{1, 2, 3\}$, dann\\
		$\mathfrak{P}(M) =\{\emptyset ,1, 2, 3, \{1, 2\}, \{1, 3\}, \{2, 3\}, \{1,2,3\}\}$, ($\emptyset$ ist \underline{immer} Element der Potenzmenge)\\
		$A\in 2^M$ ist äquivalent zu $A\subseteq M$\\
		$2^{\lvert M\rvert} = \lvert\mathfrak{P}(M)\rvert$
	\end{frame}
	\begin{frame}
		\frametitle{Große Vereinigungen und Durchschnitt}
		\dots%inhalt einfügen
	\end{frame}
	\begin{frame}
		\frametitle{Mehr zu Relationen}
		reflexiv, transitiv (homo/heterogen)
	\end{frame}
	\begin{frame}
		\frametitle{Aufgaben}
		Sei M eine Menge\\
		\textcolor{red}{Aufgabe 1} Zeige: $2^{\lvert M\rvert} = \lvert\mathfrak{P}(M)\rvert$\\
		%etwas zu großen Mengen und relationen
	\end{frame}
\end{document}