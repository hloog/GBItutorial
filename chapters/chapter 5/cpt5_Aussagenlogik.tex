\documentclass{beamer}
\usepackage[T1]{fontenc}
\usepackage[ngerman]{babel}
\usepackage[utf8]{inputenc}
\usepackage{lmodern}
\usepackage{amsmath}
\usepackage{amssymb}
\usepackage{mathrsfs}
\usepackage{color}
%
\title{Kapitel 5: Aussagenlogik}
\author{Raphael Adalid Braun, Jannis Weis}
\date{\today}
\institute{KIT - Karlruher Institut für Technologie}
\usetheme{Warsaw}
\begin{document}
	\begin{frame}
		\titlepage
	\end{frame}
	\begin{frame}
		\frametitle{Aussagenlogik}
		\underline{Zweiwertigkeit:} Jede Aussage ist entweder wahr oder falsch\\
		\underline{Aussagenlogische Konnektive}\\
		Negation: $\neg$\\
		logisches \textbf{und}: $\wedge$\\
		logisches \textbf{oder}: $\vee$\\
		logische Folgerung: $\rightarrow$ (Implikation)\\
		Aussagevariablen: $Var_{AL}$, Elemente meist mit $P, Q, R, S$ bezeichnet
		Alphabet der Aussagenlogik: $A_{AL}=\{(,),\neg,\wedge,\vee,\rightarrow\}\cup Var_{AL}$\\
	\end{frame}
	\begin{frame}
		\frametitle{Konstruktionsabbildungen}
		$f_{\neg}: A^{\ast}_{AL}\longrightarrow A^{\ast}_{AL}, G\mapsto (\neg G)$\\
		$f_{\wedge}:A^{\ast}_{AL}\times A^{\ast}_{AL}\longrightarrow A^{\ast}_{AL}, (G, H)\mapsto (G\wedge H)$\\
		$f_{\vee}:A^{\ast}_{AL}\times A^{\ast}_{AL}\longrightarrow A^{\ast}_{AL}, (G, H)\mapsto (G\vee H)$\\
		$f_{\rightarrow}:A^{\ast}_{AL}\times A^{\ast}_{AL}\longrightarrow A^{\ast}_{AL}, (G, H)\mapsto (G\rightarrow H)$\\
		Bindungsstärke:\\
		$\neg>\vee>\wedge>\rightarrow$\\
	\end{frame}
	\begin{frame}
		\frametitle{Aussagenlogischen Term konstruieren}
		Gesucht: $((A\rightarrow B)\wedge (\neg A\vee\neg C))$\\
		Vorgehen: Immer zuerst kleine Terme konstruieren und verbinden\\
		\textcolor{green}{$f_{\neg} ($}$A$\textcolor{green}{$)$}$ = \neg A$\\
		\textcolor{blue}{$f_{\neg}($}$C$\textcolor{blue}{$)$}$ = \neg C$\\
		\textcolor{red}{$f_{\vee}($}\textcolor{green}{$f_{\neg} ($}$A$\textcolor{green}{$)$}, \textcolor{blue}{$f_{\neg}($}$C$\textcolor{blue}{$)$}\textcolor{red}{$)$}$ = (\neg A\vee\neg C)$\\
		\textcolor{purple}{$f_{\rightarrow}($}$A,B$\textcolor{purple}{$)$}$ = (A\rightarrow B)$\\
		\textcolor{brown}{$f_{\rightarrow}($}\textcolor{red}{$f_{\vee}($}\textcolor{green}{$f_{\neg} ($}$A$\textcolor{green}{$)$}, \textcolor{blue}{$f_{\neg}($}$C$\textcolor{blue}{$)$}\textcolor{red}{$)$},\textcolor{purple}{$f_{\rightarrow}($}$A,B$\textcolor{purple}{$)$}\textcolor{brown}{$)$}$ = ((A\rightarrow B)\wedge (\neg A\vee\neg C))$\\
	\end{frame}
	\begin{frame}
		\frametitle{Interpretation von Formeln}
		Interpretation $I$, $I: Var_{AL}\longrightarrow\mathbb{B}$\\
		Auswertung: $val_{I}: For_{AL}\longrightarrow\mathbb{B}$\\
		Funktioniert analog wie Konstruktionsabbildungen.\\
		Bsp: Seien $P, Q, R$ Aussagen mit $I(P), I(Q) = w$ und $I(R) = f$\\
		$val_{I}((((P\rightarrow Q)\rightarrow R)\wedge(R\vee Q))\rightarrow P$)\\ %boolsche funktion noch einbringen
		$=b_{\rightarrow}(val_{I}(((P\rightarrow Q)\rightarrow R)\wedge(R\vee Q)), val_{I}(P))$\\
		$=b_{\rightarrow}(b_{\wedge}(val_{I}((P\rightarrow Q)\rightarrow R),val_{I}(R\vee Q)), I(P))$\\
		$=b_{\rightarrow}(b_{\wedge}(b_{\rightarrow}(val_{I}(P\rightarrow Q), val_{I}(R)), val_{I}(R\vee Q)), w)$\\
		$=~b_{\rightarrow}(b_{\wedge}(b_{\rightarrow}(b_{\rightarrow}(val_{I}(P), val_{I}(Q)), I(R)), b_{\vee}(val_{I}(R), val_{I}(Q))), w)$\\
		$=b_{\rightarrow}(b_{\wedge}(b_{\rightarrow}(b_{\rightarrow}(I(P), I(Q)), f), b_{\vee}(I(R), I(Q))), w)$\\
		$=b_{\rightarrow}(b_{\wedge}(b_{\rightarrow}(b_{\rightarrow}(w, w), f), b_{\vee}(f, w)), w)$\\
		$=b_{\rightarrow}(b_{\wedge}(b_{\rightarrow}(w, f), w), w)$\\
		$=b_{\rightarrow}(b_{\wedge}(f, w), w)$\\
		$=b_{\rightarrow}(f, w)$\\
		$=w$\\
	\end{frame}
	\begin{frame}
		\frametitle{Wahrheitstabellen}
		Eine Formel für alle möglichen (oder auch nur eine) Wahrheitsverteilungen auswerten\\
		Bsp: $F = (A\wedge B)\vee(B\rightarrow\neg A)$\\
		\textcolor{white}{.}\\ %zeilenumbruch/abstand schöner machen!
		\begin{tabular}{ccc|cc|c}
			$A$ & $\neg A$ & $B$ & $(A\wedge B)$ & $(B\rightarrow\neg A)$ & $F$ \\
			\hline
			$w$ & $f$ & $w$ & $w$ & $f$ & $w$\\
			$w$ & $f$ & $f$ & $f$ & $w$ & $w$\\
			$f$ & $w$ & $w$ & $f$ & $w$ & $w$\\
			$f$ & $w$ & $f$ & $f$ & $w$ & $w$\\
		\end{tabular}\\
		In diesem Fall ist $F$ sogar eine \emph{Tautologie}, d.h. alle Interpretationen von $F$ sind wahr.\\
		Zwei Formeln $G$ und $H$ heißen \emph{äquivalent}, wenn für alle Interpretationen gilt: $val_{I}(G) = val_{I}(H)$\\
		Schreibweise: $G\equiv H$\\
	\end{frame}
	\begin{frame}
		\frametitle{Aussagenlogische Regeln}
		$F,G,H\in For_{AL}$\\
		$F\wedge F=F,F\vee F=F$\\
		$F\wedge G=G\wedge F, F\vee G=G\vee F$\\
		$(F\wedge G)\vee H=(F\vee H)\wedge(G\wedge H)$\\
		$(F\vee G)\wedge H=(F\wedge H)\vee(G\vee H)$\\
		$\neg(F\vee G)=(\neg F\wedge\neg G), \neg(F\wedge G)=(\neg F\vee\neg G)$\\
		$F\wedge(F\vee G)=F, F\vee(F\wedge G)=F$\\
	\end{frame}
	%evtl modelle reinmachen? dann nochmal allg tautologie, erfüllbarkeit
	\begin{frame}
		\frametitle{Aussagenkalkül}
		\textcolor{red}{Axiome}\\
		Seien $G,H,k\in For_{AL}$\\
		$Ax_{AL1} = (G\rightarrow(H\rightarrow G))$\\
		$Ax_{AL2} = (G\rightarrow(H\rightarrow K))\rightarrow((G\rightarrow H)\rightarrow(G\rightarrow K))$\\
		$Ax_{AL3} = (\neg H\rightarrow\neg G)\rightarrow((\neg H\rightarrow G)\rightarrow H)$\\
		\textcolor{red}{Schlussregel} \emph{Modus Ponens}\\
		\emph{Aus wahrem kann man nur wahres folgern}, d.h. seien $G, H$ und $G\rightarrow H$ gegeben und sei $G$ und $G\rightarrow H$ wahr, dann muss auch $H$ wahr sein\\
		$MP : \frac{G\rightarrow H\quad G}{H}$\\
	\end{frame}
	%wahrheitstabelle machen
	\begin{frame}
		\frametitle{Aufgabe 1}
		Werten sie die Formel $F$ aus.\\
		$F = \neg((A\wedge B)\rightarrow\neg(B\vee A))$\\
		Ansatz: Zerlege Formel $F$ in kleinere Teilformeln\\
	\end{frame}
	\begin{frame}
		\frametitle{Lösung 1}
		$F = \neg((A\wedge B)\rightarrow\neg(B\vee A))$\\
		Definiere:\\
		\begin{itemize}
			\item $C:=A\wedge B$
			\item $D:=B\vee A$
			\item $E:=\neg D=\neg(B\vee A)$
			\item $G:=C\rightarrow E=(A\wedge B)\rightarrow\neg(B\vee A)$
		\end{itemize}
		$\Rightarrow F=\neg G$\\
		\begin{tabular}{cc|cccc|c}
			$A$ & $B$ & $C$ & $D$ & $E$ & $G$ & $F$\\
			\hline
			$w$ & $w$ & $w$ & $w$ & $f$ & $f$ & $w$\\
			$w$ & $f$ & $f$ & $w$ & $f$ & $w$ & $f$\\
			$f$ & $w$ & $f$ & $w$ & $f$ & $w$ & $f$\\
			$f$ & $f$ & $f$ & $f$ & $w$ & $w$ & $f$\\
		\end{tabular}
	\end{frame}
	%term per hand auswerten
	\begin{frame}
		\frametitle{Aufgabe 2}
		$F=\neg(P\rightarrow Q)\vee Q$\\
		Interpretieren sie die Formel $F$ für $I(P)=w$ und $I(Q)=f$\\
	\end{frame}
	%ebenfalls farben einfügen, wie in aufg 3
	\begin{frame}
		\frametitle{Lösung 2}
		$F=\neg(P\rightarrow Q)\vee Q$\\
		$val_{I}(\neg(P\rightarrow Q)\vee Q)$\\
		\pause
		$=b_{\vee}(val_{I}(\neg(P\rightarrow Q)), val_{I}(Q))$\\
		\pause
		$=b_{\vee}(b_{\neg}(val_{I}(P\rightarrow Q)), I(Q))$\\
		\pause
		$=b_{\vee}(b_{\neg}(b_{\rightarrow}(val_{I}(P), val_{I}(Q))), f)$\\
		\pause
		$=b_{\vee}(b_{\neg}(b_{\rightarrow}(I(P), I(Q))), f)$\\
		\pause
		$=b_{\vee}(b_{\neg}(b_{\rightarrow}(w, f)), f)$\\
		\pause
		$=b_{\vee}(b_{\neg}(f), f)$\\
		\pause
		$=b_{\vee}(w, f)$\\
		\pause
		$=w$\\
	\end{frame}
	\begin{frame}
		\frametitle{Aufgabe 3}
		\framesubtitle{Ableitung aus Prämisse}
		Prämisse: $\Gamma =\{P\rightarrow(Q\rightarrow R), Q\}$\\
		Zeige, dass $P\rightarrow R$ gilt\\
		$Ax_{AL1} = (G\rightarrow(H\rightarrow G))$\\
		$Ax_{AL2} = (G\rightarrow(H\rightarrow K))\rightarrow((G\rightarrow H)\rightarrow(G\rightarrow K))$\\
		$Ax_{AL3} = (\neg H\rightarrow\neg G)\rightarrow((\neg H\rightarrow G)\rightarrow H)$\\
		$MP : \frac{G\rightarrow H\quad G}{H}$\\
	\end{frame}
	%Lösungen bei hochladenfolie zu einer verbinden bzw einfach erste lassen und alle farbigen danach löschen
	\begin{frame}
		\frametitle{Lösung 3}
		\framesubtitle{Ableitung aus Prämisse}
		Prämisse: $\Gamma =\{P\rightarrow(Q\rightarrow R), Q\}$, zz: $P\rightarrow R$\\
		\begin{tabular}{cl}
			\textcolor{blue}{$I_{Ax_{AL2}}$} & $(P\rightarrow(Q\rightarrow R))\rightarrow((P\rightarrow Q)\rightarrow(P\rightarrow R))$\\
			\textcolor{blue}{$II_{Prämisse}$} & $P\rightarrow(Q\rightarrow R)$\\
			\textcolor{blue}{$III_{MP(I, II)}$} & $(P\rightarrow Q)\rightarrow(P\rightarrow R)$\\
			\textcolor{blue}{$IV_{Prämisse}$} & $Q$\\
			\textcolor{blue}{$V_{IV, Ax_{AL1}}$} & $Q\rightarrow(P\rightarrow Q)$\\
			\textcolor{blue}{$VI_{MP(IV, V)}$} & $P\rightarrow Q$\\
			\textcolor{blue}{$VII_{MP(III, VI)}$} & $P\rightarrow R$\\
		\end{tabular}
		$Ax_{AL1} = (G\rightarrow(H\rightarrow G))$\\
		$Ax_{AL2} = (G\rightarrow(H\rightarrow K))\rightarrow((G\rightarrow H)\rightarrow(G\rightarrow K))$\\
	\end{frame}
	%ab hier löschen -------------
	%I
	\begin{frame}
		\frametitle{Lösung 3}
		\framesubtitle{Ableitung aus Prämisse}
		Prämisse: $\Gamma =\{P\rightarrow(Q\rightarrow R), Q\}$, zz: $P\rightarrow R$\\
		\begin{tabular}{cl}
			\textcolor{green}{$I_{Ax_{AL2}}$} & $($\textcolor{red}{$P\rightarrow(Q\rightarrow R)$}$)\rightarrow((P\rightarrow Q)\rightarrow(P\rightarrow R))$\\
			\textcolor{white}{$II_{Prämisse}$} & \textcolor{white}{$P\rightarrow(Q\rightarrow R)$}\\
			\textcolor{white}{$III_{MP(I, II)}$} & \textcolor{white}{$(P\rightarrow Q)\rightarrow(P\rightarrow R)$}\\
			\textcolor{white}{$IV_{Prämisse}$} & \textcolor{white}{$Q$}\\
			\textcolor{white}{$V_{IV, Ax_{AL1}}$} & \textcolor{white}{$Q\rightarrow(P\rightarrow Q)$}\\
			\textcolor{white}{$VI_{MP(IV, V)}$} & \textcolor{white}{$P\rightarrow Q$}\\
			\textcolor{white}{$VII_{MP(III, VI)}$} & \textcolor{white}{$P\rightarrow R$}\\
		\end{tabular}
		$Ax_{AL1} = (G\rightarrow(H\rightarrow G))$\\
		$Ax_{AL2} = ($\textcolor{red}{$G\rightarrow(H\rightarrow K)$}$)\rightarrow((G\rightarrow H)\rightarrow(G\rightarrow K))$\\
	\end{frame}
	%II
	\begin{frame}
		\frametitle{Lösung 3}
		\framesubtitle{Ableitung aus Prämisse}
		Prämisse: $\Gamma =\{$\textcolor{red}{$P\rightarrow(Q\rightarrow R)$}$, Q\}$, zz: $P\rightarrow R$\\
		\begin{tabular}{cl}
			\textcolor{blue}{$I_{Ax_{AL2}}$} & $(P\rightarrow(Q\rightarrow R))\rightarrow((P\rightarrow Q)\rightarrow(P\rightarrow R))$\\
			\textcolor{green}{$II_{Prämisse}$} & \textcolor{red}{$P\rightarrow(Q\rightarrow R)$}\\
			\textcolor{white}{$III_{MP(I, II)}$} & \textcolor{white}{$(P\rightarrow Q)\rightarrow(P\rightarrow R)$}\\
			\textcolor{white}{$IV_{Prämisse}$} & \textcolor{white}{$Q$}\\
			\textcolor{white}{$V_{IV, Ax_{AL1}}$} & \textcolor{white}{$Q\rightarrow(P\rightarrow Q)$}\\
			\textcolor{white}{$VI_{MP(IV, V)}$} & \textcolor{white}{$P\rightarrow Q$}\\
			\textcolor{white}{$VII_{MP(III, VI)}$} & \textcolor{white}{$P\rightarrow R$}\\
		\end{tabular}
		$Ax_{AL1} = (G\rightarrow(H\rightarrow G))$\\
		$Ax_{AL2} = (G\rightarrow(H\rightarrow K))\rightarrow((G\rightarrow H)\rightarrow(G\rightarrow K))$\\
	\end{frame}
	%III
	\begin{frame}
		\frametitle{Lösung 3}
		\framesubtitle{Ableitung aus Prämisse}
		Prämisse: $\Gamma =\{P\rightarrow(Q\rightarrow R), Q\}$, zz: $P\rightarrow R$\\
		\begin{tabular}{cl}
			\textcolor{blue}{$I_{Ax_{AL2}}$} & $($\textcolor{gray}{$P\rightarrow(Q\rightarrow R)$}$)\rightarrow($\textcolor{red}{$(P\rightarrow Q)\rightarrow(P\rightarrow R)$}$)$\\
			\textcolor{blue}{$II_{Prämisse}$} & \textcolor{gray}{$P\rightarrow(Q\rightarrow R)$}\\
			\textcolor{green}{$III_{MP(I, II)}$} & \textcolor{red}{$(P\rightarrow Q)\rightarrow(P\rightarrow R)$}\\
			\textcolor{white}{$IV_{Prämisse}$} & \textcolor{white}{$Q$}\\
			\textcolor{white}{$V_{IV, Ax_{AL1}}$} & \textcolor{white}{$Q\rightarrow(P\rightarrow Q)$}\\
			\textcolor{white}{$VI_{MP(IV, V)}$} & \textcolor{white}{$P\rightarrow Q$}\\
			\textcolor{white}{$VII_{MP(III, VI)}$} & \textcolor{white}{$P\rightarrow R$}\\
		\end{tabular}
		$Ax_{AL1} = (G\rightarrow(H\rightarrow G))$\\
		$Ax_{AL2} = (G\rightarrow(H\rightarrow K))\rightarrow((G\rightarrow H)\rightarrow(G\rightarrow K))$\\
	\end{frame}
	%IV
	\begin{frame}
		\frametitle{Lösung 3}
		\framesubtitle{Ableitung aus Prämisse}
		Prämisse: $\Gamma =\{P\rightarrow(Q\rightarrow R), $\textcolor{red}{$Q$}$\}$, zz: $P\rightarrow R$\\
		\begin{tabular}{cl}
			\textcolor{blue}{$I_{Ax_{AL2}}$} & $(P\rightarrow(Q\rightarrow R))\rightarrow((P\rightarrow Q)\rightarrow(P\rightarrow R))$\\
			\textcolor{blue}{$II_{Prämisse}$} & $P\rightarrow(Q\rightarrow R)$\\
			\textcolor{blue}{$III_{MP(I, II)}$} & $(P\rightarrow Q)\rightarrow(P\rightarrow R)$\\
			\textcolor{green}{$IV_{Prämisse}$} & \textcolor{red}{$Q$}\\
			\textcolor{white}{$V_{IV, Ax_{AL1}}$} & \textcolor{white}{$Q\rightarrow(P\rightarrow Q)$}\\
			\textcolor{white}{$VI_{MP(IV, V)}$} & \textcolor{white}{$P\rightarrow Q$}\\
			\textcolor{white}{$VII_{MP(III, VI)}$} & \textcolor{white}{$P\rightarrow R$}\\
		\end{tabular}
		$Ax_{AL1} = (G\rightarrow(H\rightarrow G))$\\
		$Ax_{AL2} = (G\rightarrow(H\rightarrow K))\rightarrow((G\rightarrow H)\rightarrow(G\rightarrow K))$\\
	\end{frame}
	%V
	\begin{frame}
		\frametitle{Lösung 3}
		\framesubtitle{Ableitung aus Prämisse}
		Prämisse: $\Gamma =\{P\rightarrow(Q\rightarrow R), $\textcolor{red}{$Q$}$\}$, zz: $P\rightarrow R$\\
		\begin{tabular}{cl}
			\textcolor{blue}{$I_{Ax_{AL2}}$} & $(P\rightarrow(Q\rightarrow R))\rightarrow((P\rightarrow Q)\rightarrow(P\rightarrow R))$\\
			\textcolor{blue}{$II_{Prämisse}$} & $P\rightarrow(Q\rightarrow R)$\\
			\textcolor{blue}{$III_{MP(I, II)}$} & $(P\rightarrow Q)\rightarrow(P\rightarrow R)$\\
			\textcolor{blue}{$IV_{Prämisse}$} & $Q$\\
			\textcolor{green}{$V_{IV, Ax_{AL1}}$} & \textcolor{red}{$Q$}$\rightarrow(P\rightarrow $\textcolor{red}{$Q$}$)$\\
			\textcolor{white}{$VI_{MP(IV, V)}$} & \textcolor{white}{$P\rightarrow Q$}\\
			\textcolor{white}{$VII_{MP(III, VI)}$} & \textcolor{white}{$P\rightarrow R$}\\
		\end{tabular}
		$Ax_{AL1} = ($\textcolor{red}{$G$}$\rightarrow(H\rightarrow $\textcolor{red}{$G$}$))$\\
		$Ax_{AL2} = (G\rightarrow(H\rightarrow K))\rightarrow((G\rightarrow H)\rightarrow(G\rightarrow K))$\\
	\end{frame}
	%VI
	\begin{frame}
		\frametitle{Lösung 3}
		\framesubtitle{Ableitung aus Prämisse}
		Prämisse: $\Gamma =\{P\rightarrow(Q\rightarrow R), Q\}$, zz: $P\rightarrow R$\\
		\begin{tabular}{cl}
			\textcolor{blue}{$I_{Ax_{AL2}}$} & $(P\rightarrow(Q\rightarrow R))\rightarrow((P\rightarrow Q)\rightarrow(P\rightarrow R))$\\
			\textcolor{blue}{$II_{Prämisse}$} & $P\rightarrow(Q\rightarrow R)$\\
			\textcolor{blue}{$III_{MP(I, II)}$} & $(P\rightarrow Q)\rightarrow(P\rightarrow R)$\\
			\textcolor{blue}{$IV_{Prämisse}$} & \textcolor{gray}{$Q$}\\
			\textcolor{blue}{$V_{IV, Ax_{AL1}}$} & \textcolor{gray}{$Q$}$\rightarrow($\textcolor{red}{$P\rightarrow Q$}$)$\\
			\textcolor{green}{$VI_{MP(IV, V)}$} & \textcolor{red}{$P\rightarrow Q$}\\
			\textcolor{white}{$VII_{MP(III, VI)}$} & \textcolor{white}{$P\rightarrow R$}\\
		\end{tabular}
		$Ax_{AL1} = (G\rightarrow(H\rightarrow G))$\\
		$Ax_{AL2} = (G\rightarrow(H\rightarrow K))\rightarrow((G\rightarrow H)\rightarrow(G\rightarrow K))$\\
	\end{frame}
	%VII
	\begin{frame}
		\frametitle{Lösung 3}
		\framesubtitle{Ableitung aus Prämisse}
		Prämisse: $\Gamma =\{P\rightarrow(Q\rightarrow R), Q\}$, zz: $P\rightarrow R$\\
		\begin{tabular}{cl}
			\textcolor{blue}{$I_{Ax_{AL2}}$} & $(P\rightarrow(Q\rightarrow R))\rightarrow((P\rightarrow Q)\rightarrow(P\rightarrow R))$\\
			\textcolor{blue}{$II_{Prämisse}$} & $P\rightarrow(Q\rightarrow R)$\\
			\textcolor{blue}{$III_{MP(I, II)}$} & $($\textcolor{gray}{$P\rightarrow Q$}$)\rightarrow($\textcolor{red}{$P\rightarrow R$}$)$\\
			\textcolor{blue}{$IV_{Prämisse}$} & $Q$\\
			\textcolor{blue}{$V_{IV, Ax_{AL1}}$} & $Q\rightarrow(P\rightarrow Q)$\\
			\textcolor{blue}{$VI_{MP(IV, V)}$} & \textcolor{gray}{$P\rightarrow Q$}\\
			\textcolor{green}{$VII_{MP(III, VI)}$} & \textcolor{red}{$P\rightarrow R$}\\
		\end{tabular}
		$Ax_{AL1} = (G\rightarrow(H\rightarrow G))$\\
		$Ax_{AL2} = (G\rightarrow(H\rightarrow K))\rightarrow((G\rightarrow H)\rightarrow(G\rightarrow K))$\\
	\end{frame}
	%bis hier alles löschen ------------------------------------
\end{document}