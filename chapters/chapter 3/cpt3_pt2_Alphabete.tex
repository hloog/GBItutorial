\documentclass{beamer}
\usepackage[T1]{fontenc}
\usepackage[ngerman]{babel}
\usepackage[utf8]{inputenc}
\usepackage{lmodern}
\usepackage{amsmath}
\usepackage{amssymb}
\usepackage{mathrsfs}
\usepackage{color}
%
\title{Kapitel 5: Aussagenlogik}
\author{Raphael Adalid Braun, Jannis Weis}
\date{\today}
\institute{KIT - Karlruher Institut für Technologie}
\usetheme{Warsaw}
\begin{document}
	\begin{frame}
		\frametitle{Was ist ein Alphabet?}
		Alphabet = nichtleere endliche Menge von Symbolen, bspw Zahlen oder Buchstaben\\
		Beispiele:
		\begin{itemize}
			\item deutsches Alphabet
			\item $A = \{0, 1, 2, 3, 4, 5, 6, 7, 8, 9, A, B, C, D\}$
			\item ASCII und Unicode (später wichtig)
		\end{itemize}
	\end{frame}
\end{document}
