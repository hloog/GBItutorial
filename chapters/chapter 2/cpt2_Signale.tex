\documentclass{beamer}
\usepackage[T1]{fontenc}
\usepackage[ngerman]{babel}
\usepackage[utf8]{inputenc}
\usepackage{lmodern}
\usepackage{amsmath}
\usepackage{amssymb}
\usepackage{mathrsfs}
\begin{document}
	\begin{frame}
		\frametitle{Kapitel 2: Signale, Nachrichten, Informationen, Daten}
		Signal = Veränderung physikalischer Größen\\
		%bsp: licht gelangt von papier ins auge
		Nachricht =  Inschrift, zB \lq\lq{10001101001}\rq\rq auf ein Blatt Papier gemalt\\
		Information = Interpretation einer Nachricht\\
		hängt von Bezugssystem ab!\\
		zB 01001 interpretiert als Dualzahl ist 9 (Dezimalzahl)
	\end{frame}
	\begin{frame}
		\frametitle{Kapitel 2: Signale, Nachrichten, Informationen, Daten}
		\framesubtitle{Nachrichtengehalt}
		$n \equiv$ Anzahl möglicher Nachrichten
		Informationsgehalt einer Nachricht:
		\begin{description}
			\item[Naturalis](nat) $\log_e(n)$  (natural units)
			\item[Hartley] (hart) $\log_{10}(n)$ (dits [decimal digits])
			\item[Shannon] (Sh) $\log_2(n)$ (bits [binary digits])
		\end{description}
	\end{frame}
\end{document}