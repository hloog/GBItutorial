\documentclass{beamer}
\usepackage[T1]{fontenc}
\usepackage[ngerman]{babel}
\usepackage[utf8]{inputenc}
\usepackage{lmodern}
\usepackage{amsmath}
\usepackage{amssymb}
\usepackage{mathrsfs}
\usepackage{color}
%
\title{Kapitel 5: Aussagenlogik}
\author{Raphael Adalid Braun, Jannis Weis}
\date{\today}
\institute{KIT - Karlruher Institut für Technologie}
\usetheme{Warsaw}
\begin{document}
	\begin{frame}
		\frametitle{Definition von Abbildungen}
		Definitions- und Zielbereich angeben\\
		$f: A\longrightarrow B$\\
		Spezifikation von Funktionswerten (Abbildungsvorschrift)\\
		$x\mapsto$ Ausdruck, in dem x vorkommen \underline{darf}\\
		oder $f(x) :=$ Ausdruck, \dots\\
		Bsp: $f: \mathbb{N}_0\longrightarrow\mathbb{N}_0, n\mapsto
		\begin{cases}
			\frac{n}{2} & \text{, falls n gerade}\\
			3n+1 & \text{, falls n ungerade}
		\end{cases}$
	\end{frame}
	\begin{frame}
		\frametitle{Aufgaben}
		Welche dieser Abbildungen sind injektiv, welche surjektiv und welche bijektiv oder gar keine der drei Eigenschaften??\\
		\textcolor{red}{Aufgabe 1} $f:\mathbb{N}_0\longrightarrow\mathbb{N}_0, x\mapsto x+1$\\
		\textcolor{red}{Aufgabe 2} $g:\mathbb{R}_0^+\longrightarrow\mathbb{R}, x\mapsto\sqrt{\lvert x\rvert}$\\
		\textcolor{red}{Aufgabe 3} $h:\mathbb{R}\longrightarrow\mathbb{R}_0^+, x\mapsto\lvert x\rvert$\\
		\textcolor{red}{Aufgabe 3} $s:\mathbb{R}\longrightarrow\mathbb{R}^+, x\mapsto\lvert x\rvert$
		%injektivität und surjektivität zeigen / widerlegen
	\end{frame}
\end{document}
