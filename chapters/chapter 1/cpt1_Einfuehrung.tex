\documentclass[18pt]{beamer}
\usepackage{templates/beamerthemekit}
\usepackage[T1]{fontenc}
\usepackage[utf8]{inputenc}
\usepackage{libertine}
\usepackage{amsmath}
\usepackage{amssymb}
\usepackage{mathrsfs}
\usepackage{latexsym}%for \Box and \leadsto
\usepackage{mathtools} % for xRightarrow
\usepackage{tikz}%zum zusammenfassen
\usepackage{xifthen}
\newcommand{\inserttask}[2][]{\title[Aufgabe]{} %zum einfügen von Aufgaben
 \ifdefined\showSolutions
  \input{../../tasks/#2}\ifthenelse{\isempty{#1}}{}{\input{../../solutions/#1}}
 \else\input{../../tasks/#2}\fi
}
\usetikzlibrary{decorations.pathreplacing}%zum zusammenfassen

%\titleimage{mypicture}
\title[Tutorium GBI]{Tutorium 1}
\subtitle{Einführung}
\author{Raphael Adalid Braun, Jannis Weis}
\institute{Grundbegriffe der Informatik | WS 2018/19}


% Bibliography
%\usepackage[citestyle=authoryear,bibstyle=numeric,hyperref,backend=biber]{biblatex}
%\addbibresource{templates/example.bib}
%\bibhang1em

\begin{document}
\selectlanguage{ngerman}

\begin{frame}
 \titlepage
\end{frame}

\begin{frame}{Einführung}
 \framesubtitle{Parallele Tutorien}
 %add timetable with other tutorials, time and location included
 %mention that other tutorials have same content and slides
 \begin{itemize}
  \item[]Jannis Weis: uwfwp@student.kit.edu
  \item[]Raphael Adalid Braun: unvcz@student.kit.edu
  \item[]Unsere Aufgaben und Lösungen werden nicht hochgeladen, also bitte mitschreiben
 \end{itemize}
\end{frame}
\begin{frame}
 \frametitle{Einführung}
 \framesubtitle{Organisatorisches}
 \begin{itemize}
  \item Übungsblätter wöchentlich
  \item Ausgabe Mittwoch, etwa 14.00 Uhr
  \item Abgabe Freitag, 13.00 Uhr im Briefkasten im Infobau
  \item Übungsschein: $\geq50\%$ Punkte
  \item Deckblatt abgeben!
  \item Alleine abgeben, alles handschriftlich
        %date of exam?
 \end{itemize}
\end{frame}
\end{document}
