\documentclass[18pt]{beamer}
\usepackage{templates/beamerthemekit}
\usepackage[T1]{fontenc}
\usepackage[utf8]{inputenc}
\usepackage{libertine}
\usepackage{amsmath}
\usepackage{amssymb}
\usepackage{mathrsfs}
\usepackage{latexsym}%for \Box and \leadsto
\usepackage{mathtools} % for xRightarrow
\usepackage{tikz}%zum zusammenfassen
\usepackage{xifthen}
\newcommand{\inserttask}[2][]{\title[Aufgabe]{} %zum einfügen von Aufgaben
 \ifdefined\showSolutions
  \input{../../tasks/#2}\ifthenelse{\isempty{#1}}{}{\input{../../solutions/#1}}
 \else\input{../../tasks/#2}\fi
}
\usetikzlibrary{decorations.pathreplacing}%zum zusammenfassen

\title[Mengen, Alphabete...]{Kapitel 3: Mengen, Alphabete, Relationen, Abbildungen}
\subtitle{Tutorium - } %Tutoriums-Nr einfügen
\author{Raphael Adalid Braun, Jannis Weis}
\institute{Grundbegriffe der Informatik | WS 2018/19}
\titlelogo{no_logo}

\begin{document}
\selectlanguage{ngerman}

\begin{frame}
 \titlepage
\end{frame}

%Entfernen für Tutoriumsfolien-%
\def\showSolutions{1}					 %
%------------------------------%

\title[Mengen]{}
\begin{frame}{Was sind Mengen?}
	Menge = ''Behälter'' mit Objekten\\
	Elemente \textbf{können} Zahlen sein.\\
	Jedes Element kann pro Menge nur einmal vorhanden sein.\\
	\begin{equation*}
		\lbrace 1, 2, 2, 3, 3, 3\rbrace = \lbrace 1, 2, 3\rbrace
	\end{equation*}
	Wichtigste Mengen in GBI:\\
	\begin{description}
		\item[Natürliche Zahlen] $\mathbb{N}$ (Vorsicht! Unterschied zwischen $\mathbb{N}$ und $\mathbb{N}_0$)
		\item[Ganze Zahlen] $\mathbb{Z}$
		\item[Reelle Zahlen] $\mathbb{R}$
		\item[Leere Menge] $\emptyset$ oder \{   \}
	\end{description}
	$\emptyset\subseteq\mathbb{N}\subseteq\mathbb{Z}\subseteq\mathbb{R}$
\end{frame}

\begin{frame}{Abkürzungen beim Hantieren mit Mengen}
	\begin{itemize}
		\item[] Schreibweise: \textcolor{red}{x ist Element von A}, d.h. wenn x in A enthalten ist:
		      \begin{itemize}
			      \item[] \center{$x \in A$}
		      \end{itemize}
		\item[] Genauso, wenn x \textcolor{red}{nicht} in A enthalten ist:
		      \begin{itemize}
			      \item[] \center{$x \notin A$}
		      \end{itemize}
		\item[] \textcolor{red}{Für alle x}, die Element von A sind, gilt \dots
		      \begin{itemize}
			      \item[] \center{$\forall x \in A: \dots$}
		      \end{itemize}
		\item[] \textcolor{red}{Es existiert (mind. ein) x}, das Element von A ist, für das \dots gilt.
		      \begin{itemize}
			      \item[] \center{$\exists x\in A: \dots$}
		      \end{itemize}
	\end{itemize}
\end{frame}

\begin{frame}{Teilmengen}
	A und B seien Mengen, dann ist A Teilmenge von B, wenn gilt:
	\center{$\forall x\in A: x\in B$}\\
	A Teilmenge von B schreibt man:
	\center{$A\subseteq B$}\\
	A und B sind gleich ($A = B$), genau dann, wenn gilt:
	\center{$A\subseteq B$ und $B\subseteq A$}
\end{frame}

\begin{frame}{Vereinigung und Durchschnitt}
	A und B seien Mengen.\\
	\underline{Vereinigung $A\cup B$}
	\begin{itemize}
		\item $\forall x\in A\cup B: x\in A$ \textbf{oder} $x\in B$
		\item Bsp: $\{1, 2, 3\}\cup \{3, 4, 5\} = \{1, 2, 3, 4, 5\}$
	\end{itemize}
	Durchschnitt $A\cap B$
	\begin{itemize}
		\item $\forall x\in A\cap B: x\in A$ \textbf{und} $x\in B$
		\item Bsp: $\{1, 2, 3\}\cap \{3, 4, 5\} = \{3\}$
	\end{itemize}
\end{frame}

\begin{frame}{Eigenschaften von Vereinigung und Durchschnitt}
	A, B und C seien Mengen
	\begin{description}
		\item[Idempotenz] $A\cup A = A$ / $A\cap A = A$
		\item[Kommutativ] $A\cup B = B\cup A$ / $A\cap B = B\cap A$
		\item[Assoziativ] $A\cup (B\cup C) = (A\cup B) \cup C$
		\item[Distributiv] $A\cap (B\cup C) = (A\cap B)\cup (A\cap C)$ / $A\cup (B\cap C) = (A\cup B)\cap (A\cup C)$
	\end{description}
\end{frame}

\begin{frame}{Mengendifferenz und Kardinalität}
	A und B seien Mengen
	Mengendifferenz $A\setminus B$
	\begin{itemize}
		\item $\forall x\in A\setminus B: x\in A$ \textbf{und} $x\notin B$
		\item Bsp: $\{1, 2, 3, 4, 5\} \setminus \{2, 4\} = \{1, 3, 5\}$
	\end{itemize}
	Kardinalität einer Menge A = Anzahl der Elemente in A\\
	Sei $A := \{1, 2, 3\}$, dann $\lvert A\rvert = 3$, aber auch $\lvert \{1, 2, 3, 3, 3\}\rvert = 3$
\end{frame}

\begin{frame}{Aufgaben}
	Seien A,B,C Mengen, $A := \{1, 2, 3, 4, 5, 6\}, B := \{1, 4, 6, 7\}, C := \{2, 2, 3, 7\}$\\
	\textcolor{blue}{Aufgabe 1} Berechne $A\cap B, \lvert A\cap B\rvert, B\cup C, \lvert B\cup C\rvert, A\setminus C$\\
	Es sei M eine Menge und es seien $A\subseteq M$ und $B\subseteq M$\\
	\textcolor{blue}{Aufgabe 2} Zeige: $M\setminus (A\cup B) = (M\setminus A)\cap (M\setminus B)$\\
\end{frame}


\title[Alphabete]{}
\begin{frame}{Was ist ein Alphabet?}
	Alphabet = nichtleere endliche Menge von Symbolen, bspw Zahlen oder Buchstaben\\
	Beispiele:
	\begin{itemize}
		\item deutsches Alphabet
		\item $A = \{0, 1, 2, 3, 4, 5, 6, 7, 8, 9, A, B, C, D\}$
		\item ASCII und Unicode (später wichtig)
	\end{itemize}
\end{frame}

\title[Relationen]{}
\begin{frame}{Paare ($\simeq$ Tupel)}
	A und B seien Mengen, $a\in A$ und $b\in B$.
	Dann ist $(a,b)$ ein Tupel\\
	$a = b$ ist erlaubt, jedoch ist bei $ a\neq b$ die Reihenfolge entscheidend! $(1, 2) \neq (2, 1)$\\
	Relation = Menge R, $R\subseteq A\times B$
\end{frame}

\begin{frame}{Kartesisches Produkt}
	Kartesisches Produkt von Mengen A und B:\\
	$A \times B := \{ (a, b)\vert a\in A$ und $b\in B\}$
	Sei $\lvert A\rvert = n$ und $\lvert B\rvert = m$, dann $\lvert A\times B\rvert = n * m$
	% evtl kartesisches produkt vieler mengen? (indukti definiert zB)
	Kartesisches Produkt mit sich selbst:\\
	$M^{4} = M\times M\times M\times M$\\
	$(m_{1}, m_{2}, m_{3}, m_{4})\in M^{4}$ genau dann, wenn $m_{1}\in M, m_{2}\in M, m_{3}\in M$ und $m_{4}\in M$\\
	$M^{n} = M \times M\times\dots\times M\times M$
\end{frame}

\begin{frame}{Eigenschaften von Relationen}
	Seien A, B Mengen und $R\subseteq A\times B$ eine Relation\\
	\underline{linkstotal}\\
	$\forall a\in A:\exists b\in B: (a,b)\in R$\\
	%Zeichnung, ``jedes a muss verwendet werden''
	\underline{rechtseindeutig}\\
	$\forall (a, b_{1}), (a, b_{2})\in R: b_{1} = b_{2}$\\
	%für alle a gilt: max ein b pro relation + Zeichnung

	Relationen, die \emph{linkstotal} und \emph{rechtseindeutig} sind, heißen \emph{\underline{Abbildungen}}\\
	Schreibweise: $R: A\longrightarrow B$\\
	Definitionsbereich, Zielbereich %Pfeile dazufügen
\end{frame}

\begin{frame}{Eigenschaften von Abbildungen}
	\underline{linkseindeutig}(injektiv)\\
	$\forall (a_{1}, b_{1}), (a_{2}, b_{2})\in R: a_{1} \neq a_{2}\Rightarrow b_{1} \neq b_{2}$\\%Implikation erklären+zeichnung
	\underline{rechtstotal}(surjektiv)\\
	$\forall b\in B: \exists a\in A: (a, b)\in R$\\
	Eine Abbildung, die  sowohl \emph {injektiv} als auch \emph{surjekti} ist, heißt \emph{bijektiv}.
\end{frame}

%% induktive definition muss an andere Stelle, passt hier nicht!
\begin{frame}{Induktive Definition}
	Startelement, meist mit Index 0 oder 1\\
	induktive Definition dann über Induktionsschritt, wie ergibt sich (n+1)-ter Fall aus n-tem Fall\\
	Beispiel: induktive Definition von Zweiterpotenz\\
	Induktionsanfang: $Z_{0} = 1$\\
	Induktionsschritt: $\forall i\in \mathbb{N}_0 : Z_{i+1} = Z_{i} * 2$\\
	Hierbei ist dann $Z_{i} = 2^{i}$
\end{frame}

\title[Abbildungen]{}
\begin{frame}{Definition von Abbildungen}
 Definitions- und Zielbereich angeben\\
 $f: A\longrightarrow B$\\
 Spezifikation von Funktionswerten (Abbildungsvorschrift)\\
 $x\mapsto$ Ausdruck, in dem x vorkommen \underline{darf}\\
 oder $f(x) :=$ Ausdruck, \dots\\
 Bsp: $f: \mathbb{N}_0\longrightarrow\mathbb{N}_0, n\mapsto
  \begin{cases}
   \frac{n}{2} & \text{, falls n gerade}   \\
   3n+1        & \text{, falls n ungerade}
  \end{cases}$
\end{frame}

\inserttask{cpt3_pt4_A1}

\title[Mehr Mengen & Relationen]{}
\include{cpt3_pt5Mehr_Mengen_Relationen}

\end{document}
