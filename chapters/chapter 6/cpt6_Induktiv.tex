\documentclass{beamer}
\usepackage[T1]{fontenc}
\usepackage[ngerman]{babel}
\usepackage[utf8]{inputenc}
\usepackage{lmodern}
\usepackage{amsmath}
\usepackage{amssymb}
\usepackage{mathrsfs}
\usepackage{color}
\usepackage{latexsym}%for \Box and \leadsto
\usepackage{mathtools} % for xRightarrow
\usepackage{tikz}%zum zusammenfassen
\usetikzlibrary{decorations.pathreplacing}%zum zusammenfassen
%
\title{Kapitel 6: Induktives Vorgehen}
\author{Raphael Adalid Braun, Jannis Weis}
\date{\today}
\institute{KIT - Karlruher Institut für Technologie}
\usetheme{Warsaw}
\begin{document}
	\begin{frame}
		\titlepage
	\end{frame}
	\begin{frame}
		\frametitle{Induktive Beweise}
		Gegeben: Aussage $\mathcal{A}$\\
		Ziel: Zeige, dass $\mathcal{A}$ für alle Fälle wahr ist, indem man zeigt, dass $\mathcal{A}$ für einen Fall wahr ist und auch für alle darauffolgenden\\
		\underline{Beispielhaftes Vorgehen:}\\
		\underline{Beh:} Aussage $\mathcal{A}$ ist für alle $n\in\mathbb{N}_{0} wahr$\\
		\underline{Induktionsanfang(IA):} Sei $n=0$, dann $\dots\checkmark$\\
		\underline{Induktionsvoraussetzung(IV):} \emph{Die Behauptung sei wahr für ein beliebiges, aber festes $n\in\mathbb{N}_{0}$}\\
		\underline{Induktionsschritt(IS):} $n\leadsto n+1$\hfill (wichtig: IV verwenden!)\\
		\qquad $\dots\xRightarrow{\text{IV}}\dots _{\Box}$
	\end{frame}
	\begin{frame}
		%hier nur mit zahlen, geht aber auch zB über Wortlänge
		\frametitle{Induktive Beweise - Beispiel}
		\underline{Beh:} $2^{3\cdot n}-1$ ist für alle $n\in\mathbb{N}$ durch $7$ teilbar\\
		\underline{IA:} $n=1:$\\
		\qquad $2^{3\cdot 1}-1=7 \Rightarrow 7\vert 7\checkmark$\\
		\underline{IV:} \emph{Die Behauptung sei wahr für ein beliebiges, aber festes $n\in\mathbb{N}$}\\
		\underline{IS:} $n\leadsto n+1:$\\
		\qquad $2^{3\cdot(n+1)}-1 = 2^{3\cdot n+3}-1$\\
		\pause
		\qquad $=2^{3\cdot n}\cdot 2^{3}-1$\\
		\pause
		\qquad $=8\cdot 2^{3\cdot n}-1$\\
		\pause
		\qquad $=7\cdot2^{3\cdot n}+2^{3\cdot n}-1\Rightarrow 2^{3\cdot(n+1)}-1)\medspace\vert\thinspace 7\quad\Box$\\
		\begin{tikzpicture}
			\hspace{11mm}
        			\draw[decoration={brace,mirror,raise=0pt},decorate] (0,0) -- node[below=2pt] {$\vert 7$} (1,0);%raise = nach unten verschieben, mirror weil sonst nach oben
        			\hspace{16.5mm}
        			\draw[decoration={brace,mirror,raise=0pt},decorate] (0,0)--node[below=2pt]{$IV$}(1.3,0);
    		\end{tikzpicture}
	\end{frame}
	\begin{frame}
		\frametitle{Aufgabe}
		Zeige mittels vollständiger Induktion:\\
		$\forall n\in\mathbb{N}: 6^{n}-5\cdot n+4$ ist durch $5$ teilbar\\
		Tipp: quadratische Ergänzung\\
	\end{frame}
	\begin{frame}
		\frametitle{Lösung}
		\underline{Beh:} $\forall n\in\mathbb{N}: 6^{n}-5\cdot n+4$ ist durch $5$ teilbar\\
		\underline{IA:} $n=1:$\\
		\qquad$6^{1}-5\cdot 1+4=5\vert 5\checkmark$\\
		\underline{IV:} \emph{Die Behauptung sei wahr für ein beliebiges, aber festes $n\in\mathbb{N}$}\\
		\underline{IS:} $n\leadsto n+1$\\
		\qquad $6^{n+1}-5\cdot (n+1)+4=6^{n}\cdot 6-5\cdot n-1$\\
		\pause
		\qquad$=6^{n}\cdot 6-30\cdot n + 25\cdot n+24-25$\\
		\pause
		\qquad $=6^{n}\cdot 6-30\cdot n+24+25\cdot n-25$\\
		\pause
		\qquad $=6^{n}\cdot(6-5\cdot n+4)+5\cdot(5\cdot n-1)$\\
		\begin{tikzpicture}	
			\hspace{17mm}
        			\draw[decoration={brace,mirror,raise=0pt},decorate] (0,0)--node[below=2pt]{$IV$}(2.3,0);
        			\hspace{29mm}
        			\draw[decoration={brace,mirror,raise=0pt},decorate] (0,0)--node[below=2pt]{$\vert5$}(2,0);
        			\hspace{27mm}$\Box$
		\end{tikzpicture}
	\end{frame}
	\begin{frame}
		\frametitle{Induktive Definition}
		\framesubtitle{Definition}
		Ein Element wird definiert, von da aus wird induktiv jedes weitere definiert.\\
		Beispiel:\\
		\centerline{$L_{0}=\epsilon$}
		\centerline{$\forall n\in\mathbb{N}_{0}:L_{n+1}=L_{n}\cup\{wab\vert w\in L_{n}\}$}
		Hier wäre dann $L_{n}=\{(ab)^{k}\vert k\leq n,k\in\mathbb{N}_{0}\}$\\
	\end{frame}
	\begin{frame}
		\frametitle{Induktive Definition}
		\framesubtitle{Varianten}
		\begin{itemize}
			\item Induktionsanfang kann an späterer Stelle, bspw. bei $n=2$ oder $n=3$ sein.
			\item Es kann mehrere Induktionsanfänge geben, zB. für gerade und ungerade Zahlen, oder wenn man sich in $L_{n+1}$ auf $L_{n-1}$ bezieht
			\item Induktionsschritt nicht $n\leadsto n+1$, sondern zB. $n\leadsto 5\cdot n$
		\end{itemize}
		Gilt auch für induktive Beweise, bei mehrfachem Induktionsanfang reicht aber, die verschiedenen Anfangswerte zu bezeichnen, als Induktionsschritt reicht $n\leadsto n+1$
	\end{frame}
\end{document}