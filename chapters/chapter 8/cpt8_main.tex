\documentclass[18pt]{beamer}
\usepackage{templates/beamerthemekit}
\usepackage[T1]{fontenc}
\usepackage[utf8]{inputenc}
\usepackage{libertine}
\usepackage{amsmath}
\usepackage{amssymb}
\usepackage{mathrsfs}
\usepackage{latexsym}%for \Box and \leadsto
\usepackage{mathtools} % for xRightarrow
\usepackage{tikz}%zum zusammenfassen
\usepackage{xifthen}
\newcommand{\inserttask}[2][]{\title[Aufgabe]{} %zum einfügen von Aufgaben
 \ifdefined\showSolutions
  \input{../../tasks/#2}\ifthenelse{\isempty{#1}}{}{\input{../../solutions/#1}}
 \else\input{../../tasks/#2}\fi
}
\usetikzlibrary{decorations.pathreplacing}%zum zusammenfassen

\title[Codierungen]{Kapitel 8: Codierungen}
\subtitle{Tutorium - } %Tutoriums-Nr einfügen
\author{Raphael Adalid Braun, Jannis Weis}
\institute{Grundbegriffe der Informatik | WS 2018/19}
\titlelogo{no_logo}

\begin{document}
\selectlanguage{ngerman}


\begin{frame}
 \titlepage
\end{frame}

%Entfernen für Tutoriumsfolien-%
\def\showSolutions{1}					 %
%------------------------------%

\section{Codierungen}
\title[Codierungen]{}
\begin{frame}{Vorbereitung}
 \framesubtitle{mod und div}
 mod = Rest bei einer Ganzzahldivision, vergleichbar mit dem \%-Operator in Java\\
 div = Ergebnis einer Ganzzahldivision, also ohne Rest\\
 Beispiel:\\
 \center
 \begin{tabular}{c|c}
  \textbf{mod}   & \textbf{div}  \\
  \hline
  $18$mod$4=2$   & $18$div$4=4$  \\
  $19$mod$4=3$   & $19$div$4=4$  \\
  $20$mod$4=0$   & $20$div$4=5$  \\
  $37$mod$13=11$ & $37$div$13=2$ \\
 \end{tabular}
 \flushleft
 \emph{Rechenregeln}\\
 \small
 \centerline{$\forall x,y\in\mathbb{N}_{0}:x=y\cdot(x$div$y)+(x$mod$y)$}
 \centerline{$\forall x,y\in\mathbb{N}_{0}:(x\pm y)$mod$k=((x$mod$k)\pm(y$mod$k))$mod$k$}
 \normalsize
\end{frame}

\begin{frame}{Zahlendarstellung}
 Alphabet $Z_{n}=\{x\in\mathbb{N}_{0}\vert x<n\}$\\
 Funktion $Num_{k}: Z_{k}^{\ast}\longrightarrow\mathbb{N}_{0},$\\
 \small
 $Num_{k}(\epsilon)=0$\\
 $Num_{k}(x_{m}x_{m-1}\dots x_{3}x_{2}x_{1})=k\cdot Num_{k}(x_{m}x_{m-1}\dots x_{3}x_{2})+num_{k}(x_{1})$\\
 Man kürzt $Num_{k}(x)$ oft direkt zu $num_{k}(x)$, wenn $x$ in $Z_{k}$ liegt\\
 \normalsize
 Funktion $num_{k}: Z_{k}\longrightarrow\mathbb{N}_{0},$\small$\quad num_{k}(x)=x$\normalsize\\
\end{frame}

\begin{frame}{Zahlendarstellung}
 \framesubtitle{Beispiele}
 \begin{enumerate}
  \item[$\triangleright$] $Num_{10}(742)=10\cdot Num_{10}(74)+num_{10}(2)$\\
        $=10\cdot(10\cdot Num_{10}(7)+num_{10}(4))+2$\\
        $=10\cdot(10\cdot(10\cdot (Num_{10}(\epsilon))+num_{10}(7))+4)+2$\\%hier ausführlich mit epsilon, unten mit abkürzen
        $=10\cdot(10\cdot(10\cdot (0)+7)+4)+2$\\
        $=10\cdot(10\cdot(7)+4)+2$\\
        $=10\cdot(74)+2$\\
        $=742$\\
  \item[$\triangleright$] $Num_{2}(01101)=2\cdot Num_{2}(0110)+num_{2}(1)$\\
        $=2\cdot(2\cdot Num_{2}(011)+num_{2}(0))+2^{0}\cdot1$\\
        $=2\cdot(2\cdot (2\cdot Num_{2}(01)+num_{2}(1))+0)+2^{0}\cdot1$\\
        $=2\cdot(2\cdot (2\cdot (2\cdot Num_{2}(0)+num_{2}(1))+1)+0)+2^{0}\cdot1$\\
        $=2^{2}\cdot (2\cdot (2\cdot num_{2}(0)+1)+1)+2^{1}\cdot0+2^{0}\cdot1$\\
        $=2^{3}\cdot (2\cdot 0+1)+2^{2}\cdot1+2^{1}\cdot0+2^{0}\cdot1$\\
        $=2^{4}\cdot 0+2^{3}\cdot 1+2^{2}\cdot1+2^{1}\cdot0+2^{0}\cdot1$\\
        $=0+8+4+0+1=13$
 \end{enumerate}
\end{frame}

\begin{frame}{Hexadezimaldarstellung}
 Wie stellt man Basen $\geq10$ dar?\\
 $Z_{16}=\{0,1,2,3,4,5,6,7,8,9,10,11,12,13,14,15\}$?\\
 Nein, denn $10,11,12,13,14,15$ bestehen bereits aus zwei Zeichen, man weicht auf andere zeichen aus\\
 $Z_{16}=\{0,1,2,3,4,5,6,7,8,9,A,B,C,D,E,F\}$\\
 wobei $A=10,B=11,\dots$\\
\end{frame}

\begin{frame}{Zahlenrepräsentation}
 Sei $k\in\mathbb{N}_{0}$ und $k\geq2$\\
 $Repr_{k}:\mathbb{N}_{0}\longrightarrow Z_{k},$\\
 $n\mapsto~
  \begin{cases}
   repr_{k}(n)                                        & \text{,falls $n<k$}     \\
   Repr_{k}(n\text{div}k)\cdot repr_{k}(n\text{mod}k) & \text{,falls $n\geq k$} \\
  \end{cases}$\\
 \fbox{$\forall n\in\mathbb{N}_{0}:Num_{k}(Repr_{k}(n))=n$}\\
\end{frame}

\inserttask[cpt8_pt1_L1]{cpt8_pt1_A1}

\begin{frame}{Übersetzung von Zahldarstellungen}
 \framesubtitle{Kommt wieder im zweiten Semester in Digitaltechnik \& Entwurfsverfahren}
 Geht nur, wenn Basis Vielfaches von anderer ist, zB 2 von 16\\
 Beispiel: Zahlen von $Z_{2}$ nach $Z_{16}$ und zurück konvertieren. $2^{\textcolor{red}{4}}=16$\\
 Wir müssen \textcolor{red}{4} Zahlen aus der Dualzahl zusammenfassen, um die Hexadezimalzahl zu bekommen\\
 $1 0110 0101 1010_{2}=\textcolor{blue}{000}1\thinspace0110\thinspace0101\thinspace1010_{2}=165A_{16}$\par
 \begin{tikzpicture}
  \hspace{31.7mm}
  \draw[decoration={brace,mirror,raise=-4pt},decorate] (0,0) -- node[below=-2pt] {$1$} (0.75,0);
  \hspace{8.35mm}
  \draw[decoration={brace,mirror,raise=-4pt},decorate] (0,0)--node[below=-2pt]{$6$}(0.75,0);
  \hspace{8.355mm}
  \draw[decoration={brace,mirror,raise=-4pt},decorate] (0,0)--node[below=-2pt]{$5$}(0.75,0);
  \hspace{8.4mm}
  \draw[decoration={brace,mirror,raise=-4pt},decorate] (0,0)--node[below=-2pt]{$10$}(0.75,0);
 \end{tikzpicture}
 \textcolor{white}{.}\\%schöner machen!
 $C7F_{16}=1100\thinspace 0111\thinspace 1111_{2}$\\
 \begin{tikzpicture}
  \hspace{15.57mm}
  \draw[decoration={brace,mirror,raise=-4pt},decorate] (0,0) -- node[below=-2pt] {{\footnotesize$12$}} (0.75,0);
  \hspace{8.35mm}
  \draw[decoration={brace,mirror,raise=-4pt},decorate] (0,0)--node[below=-2pt]{{\footnotesize$7$}}(0.75,0);
  \hspace{8.355mm}
  \draw[decoration={brace,mirror,raise=-4pt},decorate] (0,0)--node[below=-2pt]{{\footnotesize $15$}}(0.75,0);
 \end{tikzpicture}
\end{frame}
%Rechnen in Zk (restklassen) mit zeichnung. ``vergleichbar mit restklassen aus LA1''
%Homomorphismen...
\begin{frame}{Huffmann-Codierung}
 gegeben: Alphabet $A$ und Wort $W\in A^{\ast}$\\
 \begin{itemize}
  \item[$\bullet$] $h:A^{\ast}\longrightarrow Z_{2}^{\ast}$
  \item[$\bullet$] $\epsilon$-freier Homomorphismus
 \end{itemize}
 findet kürzeste Codierung von $w$ in $Z_{2}$\\
\end{frame}
\begin{frame}
 \frametitle{Huffman-Codierung}
 \framesubtitle{Huffman-Baum}
 
\end{frame}


\end{document}
