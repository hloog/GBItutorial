\documentclass{beamer}
\usepackage[T1]{fontenc}
\usepackage[ngerman]{babel}
\usepackage[utf8]{inputenc}
\usepackage{lmodern}
\usepackage{amsmath}
\usepackage{amssymb}
\usepackage{mathrsfs}
\begin{document}
	\begin{frame}
		\frametitle{WILLKOMMEN(oder sowas in der Art)}
	\end{frame}
	\begin{frame}
		\frametitle{Einführung}
		\framesubtitle{Parallele Tutorien}
		%add timetable with other tutorials, time and location included
		%mention that other tutorials have same content and slides
		Jannis Weis: uwfwp@student.kit.edu
		Raphael Adalid Braun: unvcz@student.kit.edu
	\end{frame}
	\begin{frame}
		\frametitle{Einführung}
		\framesubtitle{Organisatorisches}
		\begin{itemize}
			\item Übungsblätter wöchentlich
			\item Ausgabe Mittwoch, etwa 14.00 Uhr
			\item Abgabe Freitag, 13.00 Uhr im Briefkasten im Infobau
			\item Übungsschein: $\geq50\%$ Punkte
			\item Deckblatt abgeben!
			\item Alleine abgeben, alles handschriftlich
			%date of exam?
		\end{itemize}
	\end{frame}
\end{document}
