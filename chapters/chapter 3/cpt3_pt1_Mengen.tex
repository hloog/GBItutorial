\documentclass{beamer}
\usepackage[T1]{fontenc}
\usepackage[ngerman]{babel}
\usepackage[utf8]{inputenc}
\usepackage{lmodern}
\usepackage{amsmath}
\usepackage{amssymb}
\usepackage{mathrsfs}
\usepackage{color}
\begin{document}
	\begin{frame}
		\frametitle{Kapitel 3: Mengen, Alphabete und Abbildungen}
		\framesubtitle{Was sind Mengen?}
		Menge = ''Behälter'' mit Objekten\\
		Elemente \textbf{können} Zahlen sein.\\
		Jedes Element kann pro Menge nur einmal vorhanden sein.\\
		\begin{equation*}
			\lbrace 1, 2, 2, 3, 3, 3\rbrace = \lbrace 1, 2, 3\rbrace
		\end{equation*}
		Wichtigste Mengen in GBI:\\
		\begin{description}
			\item[Natürliche Zahlen] $\mathbb{N}$ (Vorsicht! Unterschied zwischen $\mathbb{N}$ und $\mathbb{N}_0$)
			\item[Ganze Zahlen] $\mathbb{Z}$
			\item[Reelle Zahlen] $\mathbb{R}$
			\item[Leere Menge] $\emptyset$ oder \{   \}
		\end{description}
		$\emptyset\subseteq\mathbb{N}\subseteq\mathbb{Z}\subseteq\mathbb{R}$
	\end{frame}
	\begin{frame}
		\frametitle{Abkürzungen beim Hantieren mit Mengen}
		Schreibweise für: \textcolor{red}{x ist Element von A}, d.h. wenn x in A enthalten ist:\\
		\center{$x \in A$}\\
		Genauso $x \notin A$, wenn x \textcolor{red}{nicht} in A enthalten ist\\
		\textcolor{red}{Für alle x}, die Element von A sind, gilt \dots\\
		\center{$\forall x \in A: \dots$}
		\textcolor{red}{Es existiert (mind. ein) x}, das Element von A ist, für das \dots gilt.
		\center{$\exists x\in A: \dots$}
	\end{frame}
	\begin{frame}
		\frametitle{Teilmengen}
		A und B seien Mengen, dann ist A Teilmenge von B, wenn gilt:
		\center{$\forall x\in A: x\in B$}\\
		A Teilmenge von B schreibt man:
		\center{$A\subseteq B$}\\
		A und B sind gleich ($A = B$), genau dann, wenn gilt:
		\center{$A\subseteq B$ und $B\subseteq A$}
	\end{frame}
	\begin{frame}
		\frametitle{Vereinigung und Durchschnitt}
		A und B seien Mengen.\\
		\underline{Vereinigung $A\cup B$}
		\begin{itemize}
			\item $\forall x\in A\cup B: x\in A$ \textbf{oder} $x\in B$
			\item Bsp: $\{1, 2, 3\}\cup \{3, 4, 5\} = \{1, 2, 3, 4, 5\}$
		\end{itemize}
		Durchschnitt $A\cap B$
		\begin{itemize}
			\item $\forall x\in A\cap B: x\in A$ \textbf{und} $x\in B$
			\item Bsp: $\{1, 2, 3\}\cap \{3, 4, 5\} = \{3\}$
		\end{itemize}
	\end{frame}
	\begin{frame}
		\frametitle{Eigenschaften von Vereinigung und Durchschnitt}
		A, B und C seien Mengen
		\begin{description}
			\item[Idempotenz] $A\cup A = A$ / $A\cap A = A$
			\item[Kommutativ] $A\cup B = B\cup A$ / $A\cap B = B\cap A$
			\item[Assoziativ] $A\cup (B\cup C) = (A\cup B) \cup C$
			\item[Distributiv] $A\cap (B\cup C) = (A\cap B)\cup (A\cap C)$ / $A\cup (B\cap C) = (A\cup B)\cap (A\cup C)$
		\end{description}
	\end{frame}
	\begin{frame}
		\frametitle{Mengendifferenz und Kardinalität}
		A und B seien Mengen
		Mengendifferenz $A\setminus B$
		\begin{itemize}
			\item $\forall x\in A\setminus B: x\in A$ \textbf{und} $x\notin B$
			\item Bsp: $\{1, 2, 3, 4, 5\} \setminus \{2, 4\} = \{1, 3, 5\}$
		\end{itemize}
		Kardinalität einer Menge A = Anzahl der Elemente in A\\
		Sei $A := \{1, 2, 3\}$, dann $\lvert A\rvert = 3$, aber auch $\lvert \{1, 2, 3, 3, 3\}\rvert = 3$
	\end{frame}
	\begin{frame}
		\frametitle{Aufgaben}
		Seien A,B,C Mengen, $A := \{1, 2, 3, 4, 5, 6\}, B := \{1, 4, 6, 7\}, C := \{2, 2, 3, 7\}$\\
		\textcolor{blue}{Aufgabe 1} Berechne $A\cap B, \lvert A\cap B\rvert, B\cup C, \lvert B\cup C\rvert, A\setminus C$\\
		Es sei M eine Menge und es seien $A\subseteq M$ und $B\subseteq M$\\
		\textcolor{blue}{Aufgabe 2} Zeige: $M\setminus (A\cup B) = (M\setminus A)\cap (M\setminus B)$\\
	\end{frame}
\end{document}