\documentclass[18pt]{beamer}
\usepackage{templates/beamerthemekit}
\usepackage[T1]{fontenc}
\usepackage[utf8]{inputenc}
\usepackage{libertine}
\usepackage{amsmath}
\usepackage{amssymb}
\usepackage{mathrsfs}
\usepackage{latexsym}%for \Box and \leadsto
\usepackage{mathtools} % for xRightarrow
\usepackage{tikz}%zum zusammenfassen
\usepackage{xifthen}
\newcommand{\inserttask}[2][]{\title[Aufgabe]{} %zum einfügen von Aufgaben
 \ifdefined\showSolutions
  \input{../../tasks/#2}\ifthenelse{\isempty{#1}}{}{\input{../../solutions/#1}}
 \else\input{../../tasks/#2}\fi
}
\usetikzlibrary{decorations.pathreplacing}%zum zusammenfassen

\title[Signale, Nachrichten...]{Kapitel 2: Signale, Nachrichten, Informationen, Daten}
\subtitle{Tutorium - } %Tutoriums-Nr einfügen
\author{Raphael Adalid Braun, Jannis Weis}
\institute{Grundbegriffe der Informatik | WS 2018/19}
\titlelogo{no_logo}

\begin{document}
\selectlanguage{ngerman}

\begin{frame}
 \titlepage
\end{frame}

%Entfernen für Tutoriumsfolien-%
\def\showSolutions{1}					 %
%------------------------------%

\title[Signale]{}
\begin{frame}
	Signal = Veränderung physikalischer Größen\\
	%bsp: licht gelangt von papier ins auge
	Nachricht =  Inschrift, zB \lq\lq{10001101001}\rq\rq auf ein Blatt Papier gemalt\\
	Information = Interpretation einer Nachricht\\
	hängt von Bezugssystem ab!\\
	zB 01001 interpretiert als Dualzahl ist 9 (Dezimalzahl)
\end{frame}

\begin{frame}{Nachrichtengehalt}
	$n \equiv$ Anzahl möglicher Nachrichten
	Informationsgehalt einer Nachricht:
	\begin{description}
		\item[Naturalis](nat) $\log_e(n)$  (natural units)
		\item[Hartley] (hart) $\log_{10}(n)$ (dits [decimal digits])
		\item[Shannon] (Sh) $\log_2(n)$ (bits [binary digits])
	\end{description}
\end{frame}


\end{document}
